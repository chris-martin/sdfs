\documentclass[10pt]{article}
\usepackage{abstract}
\usepackage[hypcap]{caption}
\usepackage{color}
\usepackage{enumerate}
\usepackage{fullpage}
\usepackage{graphicx}
\usepackage{hyperref}
\usepackage{microtype}
\usepackage{amsmath}
\usepackage{amssymb}
\usepackage[bottom]{footmisc}

\nonstopmode

\hypersetup{hidelinks}

\title{Secure Distributed File System (SDFS)}
\date{}
\author{
  \begin{tabular}{c c}
    Kelsey Francis &
    Christopher Martin \\
    \small \tt{francis@gatech.edu} &
    \small \tt{chris.martin@gatech.edu}
  \end{tabular}
}

\begin{document}

\maketitle

\section{Protocol}

\subsection{Certificates}

Every node has a \texttt{pki} directory containing Java key stores:
\begin{itemize}
\item \texttt{ca-certs.jks} - Contains the certificates needed to trust the CA.
\item \textit{[your-keystore]}\texttt{.jdk} - Contains your cert (signed by the CA) and your private key.
\end{itemize}

\subsection{Messages}

Similarly to HTTP, each message consists of a header string followed by two newline characters
and an optional message body.
Headers are realitively short (no more than 6 lines), and we impose a maximum header size of 8KB as a sanity check.

\subsubsection{Get}

\textit{Get} messages are sent by clients to request a file.

\paragraph{Lines of the message header}
\begin{enumerate}
\item \texttt{get}
\item A new \text{correlation ID}
\item The name of the file
\end{enumerate}

\paragraph{Example header}
\begin{quote}
\texttt{%
get \\
c36c6d81-2051-49e5-8e17-1c746aa9533b \\
sas.txt
}
\end{quote}

\subsubsection{Put}

\textit{Put} messages are sent by both clients and servers.
When the server sends a \textit{put} (in response to a client \textit{get}),
the file body immediately follows.
When the client sends a \textit{put}, it waits to receive an \textit{ok}
response before sending the file content.

\paragraph{Lines of the message header}
\begin{enumerate}
\item \texttt{put}
\item A \text{correlation ID}. For a client-to-server message, this is a new ID.
      For a server-to-client message, this is the ID sent in the \textit{get}
      message that was used to request this file.
\item The name of the file
\item The SHA-512 checksum of the file contents, base 64
\item The file size in bytes, base 10
\end{enumerate}

\paragraph{Example header}
\begin{quote}
\texttt{%
put \\
c36c6d81-2051-49e5-8e17-1c746aa9533b \\
sas.txt \\
3DkQJagLJhzzo8JhjKV77CFP \textit{[\ldots]} \\
706120
}
\end{quote}

\subsubsection{Delegate}

\textit{Delegate} messages are sent by clients.

\paragraph{Lines of the message header}
\begin{enumerate}
\item \texttt{delegate}
\item A new \text{correlation ID}
\item The name of the file
\item The CN of the principal receiving delegated rights
\item Some non-empty space-delimited combination of \texttt{get}, \texttt{get*}, \texttt{put}, \texttt{put*}
\item The expiration time of the delegation, as a UNIX timestamp (number of milliseconds since 1970 UTC)
\end{enumerate}

\paragraph{Example header}
\begin{quote}
\texttt{%
delegate \\
c36c6d81-2051-49e5-8e17-1c746aa9533b \\
sas.txt \\
bob \\
get* put \\
1367021993012
}
\end{quote}

\subsubsection{Ok}

After the server receives a \textit{put}, it replies with an \textit{ok} message if the client
has permission to write the file.

\paragraph{Lines of the message header}
\begin{enumerate}
\item \texttt{ok}
\item The \text{correlation ID} of the message to which the server is responding
\item The name of the file for which write access was requested
\end{enumerate}

\paragraph{Example header}
\begin{quote}
\texttt{%
ok \\
c36c6d81-2051-49e5-8e17-1c746aa9533b \\
sas.txt
}
\end{quote}

\subsubsection{Unavailable}

The server responds to a \textit{get} or \textit{put} with an \textit{unavailable} message
if the requested resource is currently locked. This indicates a temporary failure condition,
and the client should wait briefly and retry.

\paragraph{Lines of the message header}
\begin{enumerate}
\item \texttt{unavailable}
\item The \text{correlation ID} of the message to which the server is responding
\item The name of the file on which some operation was requested
\end{enumerate}

\paragraph{Example header}
\begin{quote}
\texttt{%
unavailable \\
c36c6d81-2051-49e5-8e17-1c746aa9533b \\
sas.txt
}
\end{quote}

\subsubsection{Prohibited}

The server responds to a \textit{get} or \textit{put} with an \textit{prohibited} message
if the client does not have permission to perform the requested operation.

\paragraph{Lines of the message header}
\begin{enumerate}
\item \texttt{prohibited}
\item The \text{correlation ID} of the message to which the server is responding
\item The name of the file on which some operation was requested
\end{enumerate}

\paragraph{Example header}
\begin{quote}
\texttt{%
prohibited \\
c36c6d81-2051-49e5-8e17-1c746aa9533b \\
sas.txt
}
\end{quote}

\subsubsection{Nonexistent}

The server responds to a \textit{get} with a \textit{nonexistent} message
if the client requested a resource that does not exist.

\paragraph{Lines of the message header}
\begin{enumerate}
\item \texttt{nonexistent}
\item The \text{correlation ID} of the message to which the server is responding
\item The name of the file on which some operation was requested
\end{enumerate}

\paragraph{Example header}
\begin{quote}
\texttt{%
nonexistent \\
c36c6d81-2051-49e5-8e17-1c746aa9533b \\
sas.txt
}
\end{quote}

\subsubsection{Bye}

The client or server can send a \textit{bye} message as a friendly indicator
of intent to close the connection.

\paragraph{Lines of the message header}
\begin{enumerate}
\item \texttt{bye}
\end{enumerate}

\paragraph{Example header}
\begin{quote}
\texttt{%
bye
}
\end{quote}

\section{Performance}

\begin{tabular}{r|ll|ll}
        & Get time & Get speed & Put time & Put speed \\ \hline
 100 kB & & & 38.68 ms &  2.585 MB/s \\
   1 MB & & & 102.3 ms &  9.778 MB/s \\
  10 MB & & & 526.7 ms & 18.986 MB/s \\
 100 MB & & & 4.488 s  & 22.283 MB/s \\
   1 GB & & & 48.67 s  & 20.546 MB/s \\
  10 GB & & & 435.5 s  & 22.961 MB/s
\end{tabular}

\end{document}
