\documentclass[10pt]{article}
\usepackage{abstract}
\usepackage[hypcap]{caption}
\usepackage{color}
\usepackage{enumerate}
\usepackage{fullpage}
\usepackage{graphicx}
\usepackage{hyperref}
\usepackage{microtype}
\usepackage{amsmath}
\usepackage{amssymb}
\usepackage[bottom]{footmisc}

\nonstopmode

\hypersetup{hidelinks}

% title
\title{Secure Distributed File System (SDFS)}
\date{}
\author{
  \begin{tabular}{c c}
    Kelsey Francis &
    Christopher Martin \\
    \small \tt{francis@gatech.edu} &
    \small \tt{chris.martin@gatech.edu}
  \end{tabular}
}

\begin{document}

\section{Protocol}

\subsection{Certificates}

Every node has a `pki` directory containing Java key stores:
\begin{itemize}
\item \texttt{ca-certs.jks} - Contains the certificates needed to trust the CA.
\item \textit{[your-keystore]}\texttt{.jdk} - Contains your cert (signed by the CA) and your private key.
\end{itemize}

\subsection{Messages}

Similarly to HTTP, each message consists of a header string followed by two newline characters
and an optional message body.
Headers are realitively short (no more than 6 lines), and we impose a maximum header size of 8KB as a sanity check.

\subsubsection{Put}

\textit{Put} messages are sent by both clients and servers.

The message header consists of:
\begin{enumerate}
\item ``put''
\item A new \text{correlation id}
\item The name of the file
\item The SHA-512 checksum of the file contents, base 64
\item The file size in bytes, base 10
\end{enumerate}

\paragraph{Example header}
\begin{quote}
\texttt{%
put \\
dfd03c22-af62-40d4-86cb-ea43d399cd5c \\
sas.txt \\
3DkQJagLJhzzo8JhjKV77CFP \textit{[\ldots]} \\
706120
}
\end{quote}

\section{Performance}

\end{document}
